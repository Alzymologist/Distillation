\documentclass[12pt,a4paper,english]{ltxdoc}
\usepackage[T1]{fontenc} %# for better symbol handling
\usepackage{graphicx} %# For \includegraphics
\usepackage{fancyhdr} %# For fancy pagestyle
\usepackage{lastpage}
\usepackage[iso]{isodate} %# Makes \today print date in format 2022-11-20 (not November 20, 2022).
\usepackage{longtable}
\usepackage{ragged2e} %# for \begin{Center}
\usepackage[parfill]{parskip} %# Do not indent paragraphs
\usepackage[symbol*]{footmisc} %# for \footref
\usepackage[labelformat=empty,textfont=Large,textfont=bf, skip=0pt]{caption} %# for caption customization
\usepackage{hyperref} %# e-mail and page links

\oddsidemargin=-10pt
\textwidth=6.5in

\pagestyle{fancy} %# Makes heads and foots render
\lhead{\large Test report № \VAR{escape(current_day)}-\VAR{escape(widget["experiment_id"])}}
\rhead{\includegraphics[height=20pt]{../source/namelogo.pdf}}
\lfoot{Date: \hspace{14pt}\today}
\cfoot{}
\rfoot{Page \thepage\ of \pageref{LastPage}}
\renewcommand{\footrulewidth}{0.4pt}

\setlength{\doublerulesep}{\arrayrulewidth} %# Double rules look like one thick one
\linespread{1.5} %# Line spacing

\begin{document}

    %# These are comments! Bellow is example python code for use with jinja2.
    %# To make the example work, uncomment the lines below by removing the %#.
    %# %% for number in [1, 2, 3, 4, 9, "hello"]:
    %# %% print(number)
    %# %% print("\\\\")
    %# %% endfor
    
    \begin{flushright}
        Alzymologist Oy Lab $\bullet$ Karamalmintie 2 $\bullet$ Espoo 02330 $\bullet$ Finland \\
        \href{mailto:lab@zymologia.fi}{lab@zymologia.fi} \\
    \end{flushright}
    
    \begin{flushleft}
        \begin{multicols}{2}
            {\textbf{Beer sample info}} \\
            Product name: \VAR{escape(widget["product_name"])} \\
            Quantity of sample: \VAR{escape(widget["quantity_of_sample"])} \\
            Lot: \VAR{escape(widget["lot"])} \\
            Declared ABV: \VAR{escape(widget["declared_alcoholic_strength"])} \% \\
            Producer: \VAR{escape(widget["producer"])} \\
            Sampler: \VAR{escape(widget["sampler"])} \\
            Date of sampling: \VAR{escape(widget["date_of_sampling"])} \\
            Date of arrival: \VAR{escape(widget["date_of_arrival"])} \\
            Date of analysis: \VAR{escape(widget["date_of_analysis"])} \\
            Purpose of testing: \VAR{escape(widget["purpose_of_testing"])} \\

            \columnbreak

            {\textbf{Submitter info}} \\
            Submitting company: \VAR{escape(widget["submitting_company"])} \\
            Address: \VAR{escape(widget["address"])} \\
            Contact person: \VAR{escape(widget["contact_person"])} \\
            Phone, e-mail: \VAR{escape(widget["phone_and_email_of_contact_person"])} \\
        \end{multicols}
    \end{flushleft}

    \setlength\LTleft{0pt}
    \setlength\LTright{0pt}
    
    \begin{longtable}{@{\extracolsep{\fill}\hspace{\tabcolsep}} l c c r }
        \caption{Results} \\
        
        \hline \hline
        {\bf Parameter} & 
        \multicolumn{1}{c}{\bf Average value} & 
        \multicolumn{1}{c}{\bf Unit} & 
        \multicolumn{1}{c}{\bf Method}
        \\* \hline \hline \endhead
        %# LOOP for printing rows of a results table:
        %% for tag in requested_tags:
            \VAR{escape(tag)}
            & \VAR{escape(function_data[inverse_tag_names[tag]]["formatted_result"])}
            %% if tag in tags_with_uncertainty:
                \footnotemark[1]
            %% endif
            %% if tag is in tags_without_uncertainty:
                \footnotemark[2]
            %% endif
            & \VAR{escape(function_data[inverse_tag_names[tag]]["unit"])}
            & \VAR{escape(function_data[inverse_tag_names[tag]]["method"])}
            %% if tag in tags_without_accreditation:
                \footnotemark[3]
            %% endif
            \\
        %% endfor
    \end{longtable}
    
        %% if info_for_with_uncertainty:
            \footnotemark[1] {Values are expressed with expanded uncertainty U (k=2).} \\
        %% endif
        %% if info_for_without_uncertainty:
             \footnotemark[2] {Uncertainties have not been determined  due to insufficient data, 
             precision of results is presented in accordance with referred standard methods.} \\
        %% endif
        %% if info_for_not_accredited:
            \footnotemark[3] {Method is not accredited.} \\
        %% endif
        \\
        Analyses were carried out by \VAR{escape(widget["analyst_name"])}, analytical services chemist. \\
        Results were approved by V. Abramova, CTO of Alzymologist Oy. \\
        The results of the analysis are representative of the received sample only. \\
        The report may not be reproduced without permission from Alzymologist Oy. \\
        The report has been sent\VAR{escape(widget["results_sending_method"])}. \\
    \begin{Center}
        \emph{The end of the test report}
    \end{Center}

\end{document}
